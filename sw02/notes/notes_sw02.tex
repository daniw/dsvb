\documentclass[a4,paper,fleqn]{article}

\usepackage{../../layout/layout}

\title{Notizen DSVB -- SW02}
\date{\today}
\author{Daniel Winz}

\begin{document}
\maketitle
\clearpage

\section{Quantisierungsmodell}
\[ SNR_{\si{\deci\bel}} = 10 \cdot \log_{10} = \left(\frac{P_X}{P_\epsilon}\right) = 6 \cdot W + \delta \approx 6 \cdot W \]
\begin{itemize}
    \item Nutzsignal harmonisches Sinussignal
    \item Nutzsignal vollständig ausgesteuert
\end{itemize}
$\to$ Mit jedem zusätzlichen Bit wird der Signal-Rausch-Abstand um 6\si{\deci\bel} erhöht. 

\section{Matlab Befehle}
\begin{table}[h!]
    \begin{zebratabular}{ll}
        \rowcolor{gray} Befehl & Funktion \\
        \verb!load(spf1);! & Variable spf1 laden \\
        \verb!sound(spf1);! & Daten abspielen \\
        \verb!sound(spf1,Fs);! & Daten abspielen mit Abtastrate \verb!Fs! \\
    \end{zebratabular}
\end{table}

\section{Quantisierung in Matlab}
\begin{tabular}{ccc}
$x[n]$   & \verb!X.XXXXXX!                 & $[-1,+1)$ \\\\
         & $\Downarrow \cdot 2^{(W - 1)}$  & \\\\
         & \verb!XXXX.XXX!                 & $[-1,+1)$ \\\\
         & $\Downarrow$ \verb!round()!     & \\\\
         & \verb!XXXX.   !                 & $[-1,+1)$ \\\\
         & $\Downarrow \cdot 2^{-(W - 1)}$ & \\\\
$X_q[n]$ & \verb!X.XXX   !                 & $[-1,+1)$ \\\\
\end{tabular}

\section{Runden in Matlab}
\verb!ceil! $\rightarrow$ $\rightarrow$ \\
\verb!floor! $\leftarrow$ $\leftarrow$ \\
\verb!round! $\leftarrow$ $\rightarrow$ \\
\verb!fix! $\rightarrow$ $\leftarrow$ \\

\end{document}

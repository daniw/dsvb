\documentclass[a4,paper]{article}

\usepackage{../../layout/layout}

\title{Notizen DSVB -- SW01}
\date{\today}
\author{Daniel Winz}

\begin{document}
\maketitle
\clearpage

\section{Übergang Analog $\to$ Digital}
\[ x(t) = \sin(\omega\cdot t) \]
\[ x[n] = \sin(2 \pi \cdot f \cdot n \cdot T_s) \]
\begin{table}[h!]
    \centering
    \begin{zebratabular}{llll}
        \rowcolor{gray}
                        & analog            & Abtastung                 & digital \\
        Zeit            & $t[s]$            & $\frac{1}{T_s}[sample/s]$ & $n[sample]$ \\
        Periode         & $T[s/cycle]$      & $\frac{1}{T_s}[sample/s]$ & $\frac{T}{T_s}[sample/cycle]$ \\
        Frequenz        & $f[cycle/s = Hz]$ & $\frac{1}{f_s}[s/sample]$ & $\frac{f}{f_s}[cycle/sample]$ \\
        Kreisfrequenz   & $f[rad/s]$        & $\frac{1}{f_s}[s/sample]$ & $\frac{f}{f_s}[rad/sample = \Omega = 2 \pi \frac{f}{f_s}]$ \\
    \end{zebratabular}
    \caption{Übergang Analog $\to$ Digital}
    \label{tab:a/d}
\end{table}

\section{Korrelation}
Länge der linearen Korrelation: 
\[ \boxed{-N_x + 1 \leq n \leq N_y - 1} \]

\section{Faltung}
Länge der Faltung
\[ \boxed{0 \leq n \leq N_x + N_y - 2} \]
\end{document}


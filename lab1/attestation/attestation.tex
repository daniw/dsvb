\documentclass[a4,paper,fleqn]{article}

\usepackage{../../layout/layout}

\title{DSVB -- Testat}
\date{\today}
\author{Daniel Winz\\Yves Studer}

\fancyhead[R]{Daniel Winz\\Yves Studer} % Kopfzeile rechts
\fancyfoot[L]{Testat DSVB}       % Fusszeile links

\begin{document}
\maketitle
\clearpage

\section{Aufgaben}
\begin{enumerate}
    \item Machen Sie sich mit dem DTMF Verfahren vertraut (siehe TI\_DTMF.pdf)
    \item Analysieren Sie das Projekt \verb?App14_DTMW_ToneGen? und erledigen 
        Sie \verb?ToDo 1a,b?. Vergleichen Sie anschliessend ihren DTMF 
        Tongenerator gegen einen 
        \href{http://www.audiocheck.net/audiocheck_dtmf.php}{anderen Generator}.
        \lstsettingc
        \lstinputlisting[linerange={58-71, firstnumber=58}]{../App14_DTMF_ToneGen/Headers/DTMF.h}
        \lstsettingc
        \lstinputlisting[linerange={65-70, firstnumber=65}]{../App14_DTMF_ToneGen/Sources/sinewaves.c}
    \item In \verb?App15_DTMF_Detect? vervollständigen Sie \verb?ToDo 2?. 
        Testen Sie die Funktionalität des DTMF Detektors mit der Filter 
        Implementierung \verb?v0? und der Leistungsberechnung \verb?v0? mit 
        einem DTMF Generator (Eingang via Line-In oder Mikrofon). Passen Sie 
        den Signalpegel an bis der Detektor zuverlässig arbeitet. 
        \lstsettingmatlab
        \lstinputlisting[linerange={1-16, firstnumber=1}]{../goertzel_coeff.m}
        \lstsettingc
        \lstinputlisting[linerange={56-78, firstnumber=56}]{../App15_DTMF_Detect/Headers/goertzel.h}
    \item Siehe \nameref{q1}
    \item Vervollständigen Sie \verb?Todo 3? und implementieren Sie \verb?v1? 
        des Filters so, dass die globale Variable \verb?delay[]? nur noch zwei 
        Elemente benötigt. 
        \lstsettingc
        \lstinputlisting[linerange={86-116, firstnumber=86}]{../App15_DTMF_Detect/Sources/goertzel.c}
    \item Siehe \nameref{q2}
    \item Siehe \nameref{q3}
    \item Vervollständigen Sie \verb?ToDo 4? und \verb?ToDo 5? und testen Sie 
        ihre Leistungsberechnung Methode \verb?v1?.
        \lstsettingmatlab
        \lstinputlisting[linerange={18-31, firstnumber=18}]{../goertzel_coeff.m}
        \lstsettingc
        \lstinputlisting[linerange={79-99, firstnumber=79}]{../App15_DTMF_Detect/Headers/goertzel.h}
        \lstsettingc
        \lstinputlisting[linerange={167-202, firstnumber=167}]{../App15_DTMF_Detect/Sources/goertzel.c}
    \item Siehe \nameref{q4}
\end{enumerate}

\clearpage

\section{Fragen}

\subsection{Q1}
\label{q1}
\emph{Weshalb wird das Produkt der zwei Werte im Q15-Format}
\verb?(short int)?
\emph{in der Funktion}
\verb?goertzel_filter_v0?
\emph{mit 14 anstelle von 15 Bit skaliert? }
\paragraph{Lösung: }
Die verwendeten Koeffizienten für den Goertzelalgorithmus sind $\frac{a}{2}$. 
Die Multiplikation mit wird mit dem reduzierten Schiebebefehl realisiert. 
Dadurch kann Rechenaufwand gespart werden. 
\subsection{Q2}
\label{q2}
\emph{Warum spart}
\verb?v1?
\emph{nicht nur Speicher, sondern reduziert auch die 
Berechnungszeit?}
\paragraph{Lösung: }
Lokale Variable \verb?sum? anstelle von globaler Variable \verb?delay[0]?. Der 
Compiler kann diese in einem CPU Register anlegen. Damit kann die Zugriffszeit 
reduziert werden. 

\subsection{Q3}
\label{q3}
\emph{Ausgehend von der Gleichung (4.47) im Skript, zeige analytisch, dass die 
in der Funktion}
\verb?goertzel_output_power_v0?
\emph{verwendete Methode um die Signalleistung zu berechnen identisch zur 
Methode in Abbildung 4.17 ist, abgesehen vom endgültigen Skalierungsfaktor}
$\frac{2}{N^2}$.
\paragraph{Lösung: }
Gleichung (4.47):
\[
    P_k =
    \frac{2}{N^2} \cdot
    \left(
        \Re(Y[k])^2 + \Im(Y[k])^2
    \right)
    \qquad (4.47)
\]
\[
\Re(X[k]) = \Re(s[n] - W^k_N \cdot s[n-1])
\]
\[
\Re(X[k]) = s[n] - \cos\left(-\frac{2 \pi\cdot f_k}{f_s}\right)\cdot s[n-1]
\]
\[
\Im(X[k]) = \Im(s[n] - W^k_N \cdot s[n-1])
\]
\[
\Im(X[k]) = -\sin\left(-\frac{2 \pi\cdot f_k}{f_s}\right)\cdot s[n-1]
\]
\[
        \Re(Y[k])^2 + \Im(Y[k])^2 = \left(s[n] - \cos\left(-\frac{2 \pi\cdot f_k}{f_s}\right)\cdot s[n-1] \right)^2 + \left( -\sin\left(-\frac{2 \pi\cdot f_k}{f_s}\right)\cdot s[n-1]\right)^2
\]
\[
 = s[n]^2 - 2\cos\left(-\frac{2\pi \cdot f_k}{f_s}\right)\cdot s[n]\cdot s[n-1] + \cos^2\left(-\frac{2\pi\cdot f_k}{f_s}\right)\cdot s^2[n-1] + \sin^2\left(-\frac{2\pi\cdot f_k}{f_s}\right)\cdot c^2[n-1]
\]
\[
 = s[n]^2 - 2\cos\left(-\frac{2\pi \cdot f_k}{f_s}\right)\cdot s[n]\cdot s[n-1] + s^2[n-1]
\]
\[
 = s[n]^2 - 2\cos\left(\frac{2\pi \cdot f_k}{f_s}\right)\cdot s[n]\cdot s[n-1] + s^2[n-1]
\]
Diese Formel ist im Goerzel v0 implementiert.
%\[
%    P_k =
%    \frac{2}{N^2} \cdot
%    \left|
%        Y[k]
%    \right|^2
%\]
%\[
%    y_k[n] |_{n = N} = X[k]
%\]
%???
%\[
%    P_k =
%    \frac{2}{N^2} \cdot
%    \left|
%        y_k[n]
%    \right|^2
%\]
%\[
%    y_k[N] = s[n] - {W_N}^{k} \cdot s[n-1]
%\]
%\[
%    P_k =
%    \frac{2}{N^2} \cdot
%    \left|
%        s[n] - {W_N}^{k} \cdot s[n-1]
%    \right|^2
%\]
%\[
%    P_k =
%    \frac{2}{N^2} \cdot
%    \left(
%        s^2[n] +
%        {W_N}^{2k} \cdot s^2[n-1] -
%        2 \cdot {W_N}^{k} \cdot s[n] \cdot s[n-1]
%    \right)
%\]
%\[
%    P_k =
%    \frac{2}{N^2} \cdot
%    \left(
%        s^2[n] +
%%        e^{-\frac{4 \pi j k}{N}} \cdot s^2[n-1] -
%        2 \cdot e^{-\frac{2 \pi j k}{N}} \cdot s[n] \cdot s[n-1]
%    \right)
%\]
%???
%\[
%    P_k =
%    \frac{2}{N^2} \cdot
%    \left(
%        s^2[n]
%        + s^2[n-1]
%        - s[n] \cdot s[n-1]
%    \right)
%\]

\subsection{Q4}
\label{q4}
\emph{Vergleichen Sie die Methoden zur Leistungsberechnung}
\verb?v0?
\emph{und}
\verb?v1?
\emph{bezüglich Rechenaufwand und numerischer Robustheit. Welche Version 
würden Sie bevorzugen?}
\paragraph{Lösung: }

\end{document}

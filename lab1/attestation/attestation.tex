\documentclass[a4,paper,fleqn]{article}

\usepackage{../../layout/layout}

\title{DSVB -- Testat}
\date{\today}
\author{Daniel Winz\\Yves Studer}

\fancyhead[R]{Daniel Winz\\Yves Studer} % Kopfzeile rechts
\fancyfoot[L]{Testat DSVB}       % Fusszeile links

\begin{document}
\maketitle
\clearpage

\section{Fragen}

\subsection{Q1}
\label{q1}
\emph{Weshalb wird das Produkt der zwei Werte im Q15-Format}
\verb?(short int)?
\emph{in der Funktion}
\verb?goertzel_filter_v0?
\emph{mit 14 anstelle von 15 Bit skaliert? }
\paragraph{Lösung: }
Die verwendeten Koeffizienten für den Goertzelalgorithmus sind $\frac{a}{2}$. 
Die Multiplikation mit wird mit dem reduzierten Schiebebefehl realisiert. 
Dadurch kann Rechenaufwand gespart werden. 
\subsection{Q2}
\label{q2}
\emph{Warum spart}
\verb?v1?
\emph{nicht nur Speicher, sondern reduziert auch die 
Berechnungszeit?}
\paragraph{Lösung: }
Lokale Variable \verb?sum? anstelle von globaler Variable \verb?delay[0]?. Der 
Compiler kann diese in einem CPU Register anlegen. Damit kann die Zugriffszeit 
reduziert werden. 

\subsection{Q3}
\label{q3}
\emph{Ausgehend von der Gleichung (4.47) im Skript, zeige analytisch, dass die 
in der Funktion}
\verb?goertzel_output_power_v0?
\emph{verwendete Methode um die Signalleistung zu berechnen identisch zur 
Methode in Abbildung 4.17 ist, abgesehen vom endgültigen Skalierungsfaktor}
$\frac{2}{N^2}$.
\paragraph{Lösung: }

\subsection{Q4}
\label{q4}
\emph{Vergleichen Sie die Methoden zur Leistungsberechnung}
\verb?v0?
\emph{und}
\verb?v1?
\emph{bezüglich Rechenaufwand und numerischer Robustheit. Welche Version 
würden Sie bevorzugen?}
\paragraph{Lösung: }

\end{document}

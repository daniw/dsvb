\documentclass[a4,paper,fleqn]{article}

\usepackage{../../layout/layout}

\title{DSVB -- Testat}
\date{\today}
\author{Daniel Winz\\Yves Studer}

\fancyhead[R]{Daniel Winz\\Yves Studer} % Kopfzeile rechts
\fancyfoot[L]{Testat DSVB}       % Fusszeile links

\begin{document}
\maketitle
\clearpage

\section{Fragen}

\subsection{Q1}
\label{q1}
\emph{Why is the product of two values in Q15-format (short int) in function 
}\verb?goertzel_filter_v0?\emph{ scaled by 14 instead of 15 bits?}\\
Die verwendeten Koeffizienten für den Goertzelalgorithmus sind bereits 
$\frac{a}{2}$. 

\subsection{Q2}
\label{q2}
\emph{Why does v1 not only saves storage but also reduces the computation 
time?}\\
Lokale Variable \verb?sum? anstelle von globaler Variable \verb?delay[0]?. Der 
Compiler kann diese in einem CPU Register anlegen. Damit kann die Zugriffszeit 
reduziert werden. 

\subsection{Q3}
\label{q3}
\emph{Starting from formula (4.47) in the script, show analytically that the 
method used to calculate signal power in function 
}\verb?goertzel_output_power_v0?\emph{ is identical to the one shown in Figure 
4.17, except the final scaling factor} $\frac{2}{N^2}$.\\

\subsection{Q4}
\label{q4}
\emph{Compare power calculation methods v0 and v1 w.r.t. computational effort 
and numerical robustness. Which version would you prefer?}\\

\end{document}
